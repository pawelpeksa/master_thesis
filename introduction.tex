\section{Wprowadzenie}
\subsection{Przedmiot pracy}
Uczenie maszynowe jest dziedziną informatyki zajmującą się tworzeniem systemów zdolnych do uczenia się. Należy jednak pamiętać,  że nie jest to nauka, która przebiega tak samo jak w przypadku człowieka. Systemy samouczące zaprogramowane są do wykonywania konkretnego zadania i nie są zdolne wyjść poza te ramy. 



 Do tej pory powstało wiele algorytmów, które pozwalają na stworzenie programów komputerowych, które są w stanie uczyć się na różne sposoby. Wyróżnia się przede wszystkim dwa podstawowe rodzaje algorytmów uczenia maszynowego:
\begin{itemize}
\item uczenie z nauczycielem (uczenie nadzorowane)
\item uczenie bez nauczyciela (uczenie nienazdorowane)
\end{itemize}
Niniejsza praca skupia się na uczeniu maszynowym z nauczycielem. W celu przeprowadzeniu takiego uczenia istnieje potrzeba posiadania danych dzięki którym dany algorytm stopniowo staje się zdolny do rozwiązania zadanego mu problemu.

W kolejnych rozdziałach zostały opisane niezbędne podstawy teoretyczne bla bla
\subsection{Problematyka pracy}
Temat niniejszej pracy mówi o uczeniu maszynowym przy niskiej statystyce próbek danych. W pierwszym momencie mogłoby się zdawać, że we wspołczesnym świecie nie spotkamy się już z tym problemem. Ilość danych składowanych przez człowieka wzrasta w tempie eksponencjalnym. W wielu spotykanch problemach mamy doczynienia z danym liczącym tysiące, dziesiątki tysięcy czy nawet miliony rekordów. Istnieją jednak sytuacje w których ilość danych może stanowić problem. Wspomnieć można przynajmniej kilka takich dziedzin:
\begin{itemize}
\item medycyna - w tym przypadku możemy mieć doczynienia z sytuacją, kiedy dane pochodzą z pewnego rodzaju badania, które nie zostało do tej pory wykonane na wielu pacjentach i/lub jego wykonanie nie jest proste oraz powszechne
\item specjalistyczne zastosowania biznesowe
\item rzadko występujące zjawiska np. trzęsienia ziemi, powodzie
\item szeregi czasowe  zwłaszcza w sytuacji gdy mamy doczynienia z szeregiem czasowym o dużym okresie np. miesiąc czy rok

%źródło: https://medium.com/rants-on-machine-learning/what-to-do-with-small-data-d253254d1a89
\end{itemize}
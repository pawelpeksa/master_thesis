\section{Wprowadzenie}\label{intro}
\subsection{Przedmiot pracy}\label{matter}
% http://www.cs.put.poznan.pl/jstefanowski/ml/NEWML_W1lastr.pdf
% https://en.wikipedia.org/wiki/Computational_intelligence#Difference_between_Computational_and_Artificial_Intelligence
% https://books.google.ch/books?hl=en&lr=&id=IZosIcgJMjUC&oi=fnd&pg=PR7&dq=Computational+intelligence&ots=Dwfvy8GnKp&sig=fxvdqBnSEo4YaAU9S3q9XaUUbas#v=onepage&q=Computational%20intelligence&f=false
Uczenie maszynowe jest dziedziną informatyki zajmującą się tworzeniem systemów zdolnych do~uczenia się. Pozwala to na~wykonywanie konkretnego zadania przez~dany system bez konieczności jednoznacznego programowania go. 
Często określane jest jako obszar sztucznej inteligencji lub~też inteligencji obliczeniowej\cite{stefanowski}. Obecnie znajduje zastosowanie w~wielu dziedzinach naszego codziennego życia z~czego często nawet możemy nie zdawać sobie sprawy.
Najpopularniejsze obszary wykorzystania uczenia maszynowego pojawiają się tam, gdzie standardowe algorytmy zawodzą. Wielokrotnie związane jest to tez z~trudnością matematycznego zdefiniowania problemu. 
 Do~tej pory powstało wiele algorytmów, które~pozwalają na~tworzenie programów komputerowych, które~są w~stanie uczyć się na~różne sposoby. Wyróżnia się przede wszystkim dwa podstawowe rodzaje algorytmów uczenia maszynowego:
\begin{itemize}
\item uczenie z nauczycielem (uczenie nadzorowane),
\item uczenie bez nauczyciela (uczenie nienadzorowane).
\end{itemize}
Temat pracy mówi o~uczeniu maszynowym przy niskiej statystyce próbek danych. Implikuje to fakt, że~autor skupia się na~uczeniu maszynowym z~nauczycielem. W~celu przeprowadzeniu takiego uczenia istnieje potrzeba posiadania danych dzięki którym~dany algorytm stopniowo staje się zdolny do~rozwiązania powierzonego mu zadania..
 W~pierwszym momencie mogłoby się zdawać, że~we współczesnym świecie nie spotkamy się już z~tym problemem. Ilość danych składowanych przez~człowieka wzrasta w~tempie eksponencjalnym. W~wielu spotykanych zagadnieniach mamy do~czynienia z~danym liczącym tysiące, dziesiątki tysięcy czy~nawet miliony rekordów. Istnieją jednak sytuacje w~których można spotkać się z~niską ilością próbek danych. Wspomnieć można przynajmniej kilka takich dziedzin:
\begin{itemize}
\item medycyna - w tym przypadku możemy mieć do czynienia z sytuacją, kiedy dane pochodzą z pewnego rodzaju badania, które nie zostało do tej pory wykonane na wielu pacjentach i/lub jego wykonanie nie jest proste oraz powszechne
\item specjalistyczne zastosowania biznesowe
\item rzadko występujące zjawiska np. trzęsienia ziemi, powodzie
\item szeregi czasowe  zwłaszcza w sytuacji gdy mamy do czynienia z szeregiem o dużym kroku czasowym.
\end{itemize}

\subsection{Cel pracy}
Niniejsza praca przedstawia kilka przykładów, które~obrazują próbę poprawy jakości uczenia maszynowego w~zależności od~ilości dostępnych danych. Testom poddane  zostały następujące algorytmy: sztuczne sieci neuronowe (również z~użyciem uczenia głębokiego - \textit{deep learning}), maszyna wektorów wspierających, drzewo decyzyjne oraz~las drzew decyzyjnych. W~żadnym wypadku praca nie wyczerpuje zadanego tematu. Autor starał się określić stopień w~którym optymalizacja przy niskiej ilości danych może pomóc w~realizacji konkretnego zadania.

\subsection{Zawartość dokumentu}
Po~krótkim wprowadzeniu, w~rozdziale \ref{theory}, następuje opis podstaw teoretycznych, które~pozwalają zrozumieć postawione zadanie oraz~proponowane rozwiązania problemu. Rozdział \ref{tools} traktuje o~stworzonych systemach, wykorzystanych narzędziach oraz~organizacji pracy. W~rozdziałach \ref{results} i~\ref{conclusion} przedstawione zostały kolejno wyniki oraz~wnioski wyciągnięte na~podstawie wyników, przeprowadzonych obliczeń oraz~doświadczeń autora zdobytych podczas tworzenia kolejnych systemów uczenia maszynowego.

%źródło: https://medium.com/rants-on-machine-learning/what-to-do-with-small-data-d253254d1a89



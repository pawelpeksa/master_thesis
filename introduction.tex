\section{Wprowadzenie}
\subsection{Przedmiot pracy}
% http://www.cs.put.poznan.pl/jstefanowski/ml/NEWML_W1lastr.pdf
% https://en.wikipedia.org/wiki/Computational_intelligence#Difference_between_Computational_and_Artificial_Intelligence
% https://books.google.ch/books?hl=en&lr=&id=IZosIcgJMjUC&oi=fnd&pg=PR7&dq=Computational+intelligence&ots=Dwfvy8GnKp&sig=fxvdqBnSEo4YaAU9S3q9XaUUbas#v=onepage&q=Computational%20intelligence&f=false
Uczenie maszynowe jest dziedziną informatyki zajmującą się tworzeniem systemów zdolnych do uczenia się. Pozwala to na wykonywanie konkretnego zadania przez dany system bez koniecznosci jednoznacznego programowania go. 
Często określane jest jako obszar sztucznej inteligencji lub tez inteligencji obliczeniowej. Obecnie znajduje zastosowanie w wielu dziedzinach naszego codziennego życia z czego często nawet możemy nie zdawać sobie sprawy.
Najpopluarniejsze obszary wykorzystania uczenia maszynowego pojawiaja sie tam, gdzie standardowe algorytmy zawodza. Wielokrotnie zwiazane jest to tez z trudnoscia matematycznego zdefiniowania problemu. 
 Do tej pory powstało wiele algorytmów, które pozwalają na tworzenie programów komputerowych, które są w stanie uczyć się na różne sposoby. Wyróżnia się przede wszystkim dwa podstawowe rodzaje algorytmów uczenia maszynowego:
\begin{itemize}
\item uczenie z nauczycielem (uczenie nadzorowane)
\item uczenie bez nauczyciela (uczenie nienazdorowane)
\end{itemize}
Temat pracy mówi o uczeniu maszynowym przy niskiej statystyce próbek danych. Implikuje to fakt, ze autor skupia się na uczeniu maszynowym z nauczycielem. W celu przeprowadzeniu takiego uczenia istnieje potrzeba posiadania danych dzięki którym dany algorytm stopniowo staje się zdolny do rozwiązania zadanego mu problemu.
 W pierwszym momencie mogłoby się zdawać, że we wspołczesnym świecie nie spotkamy się już z tym problemem. Ilość danych składowanych przez człowieka wzrasta w tempie eksponencjalnym. W wielu spotykanch problemach mamy doczynienia z danym liczącym tysiące, dziesiątki tysięcy czy nawet miliony rekordów. Istnieją jednak sytuacje w których ilość danych może stanowić problem. Wspomnieć można przynajmniej kilka takich dziedzin:
\begin{itemize}
\item medycyna - w tym przypadku możemy mieć doczynienia z sytuacją, kiedy dane pochodzą z pewnego rodzaju badania, które nie zostało do tej pory wykonane na wielu pacjentach i/lub jego wykonanie nie jest proste oraz powszechne
\item specjalistyczne zastosowania biznesowe
\item rzadko występujące zjawiska np. trzęsienia ziemi, powodzie
\item szeregi czasowe  zwłaszcza w sytuacji gdy mamy doczynienia z szeregiem czasowym o dużym okresie np. miesiąc czy rok (w ktorym wartosc jest mierzona w duzych odstepach czasowych np. miesiac czy rok)

%źródło: https://medium.com/rants-on-machine-learning/what-to-do-with-small-data-d253254d1a89

W kolejnych rozdziałach zostały opisane niezbędne podstawy teoretyczne, które pozwalaja zrozumiec postawione zadanie oraz proponowane rozwiazania. W rodzialach \ref{nana} przedstawione zostaly kolejno wyniki oraz wnioski wyciagniete 
na podstawie wynikow, przeprowadzonych obliczen oraz doswiadczen autora zdobytych podczas tworzenia kolejnych programow.


\subsection{Problematyka i cel pracy}

\end{itemize}
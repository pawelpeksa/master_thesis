\section{Wykorzystane narzedzia i technologie} \label{tools}
W~celu przeprowadzenia zaproponowanych eksperymentów należało wybrać język programowania wraz z~odpowiednimi bibliotekami zapewniającymi odpowiednie funkcjonalności. Niniejszy rozdział traktuje o~technologiach wybranych celu realizacji postawionych zadań.
\subsection{Język programowania oraz biblioteki programistyczne}
Językiem programowania, który~został wybrany do~realizacji wszystkich zadań w~tej pracy został język wysokiego poziomu jakim jest Python. Jest to obecnie jeden ze~standardów w~świecie inżynierii komputerowej obok takich rozwiązań jak~np. pakiet obliczeniowy Matlab, który~jednak nie jest rozpowszechniany na~licencji \textit{open-source}. Mogłoby się wydawać na~początku, biorąc pod~uwagę, że~jest to język skryptowy, że~działanie stworzoynych programów przy jego użyciu będzie skutkowało względnie wysokim czasem wykonania. Istniejące jednak moduły takiej jak~NumPy czy~SciPy\cite{scipy} zaimplementowane są w~języku C dzięki czemu, pomimo łatwości tworzenia nowego kodu, nie tracimy jeśli chodzi o~szybkość wykonania. Dodatkowo, razem z~wymienionymi pakietami, powszechnie wykorzystywana jest biblioteka Matplotlib\cite{matplotlib} zapewniająca funkcjonalności związane z~rysowaniem wykresów. Istnieje również duża społeczność skoncentrowana wokół języka Python dzięki czemu rozwiązywanie napotkanych problemów jest dużo prostsze. Oprócz wspomnianych pakietów, wykorzystane również zostały biblioteki bezpośrednio związane z~uczeniem maszynowym:
\begin{itemize}
\item Scikit-Learn\cite{scikit} - algorytmy uczenia maszynowego oraz pokrewne (jak np. związane z ekstrakcją cech),
\item TensorFlow\cite{tensorflow} - algorytmy uczenia maszynowego wraz z głębokimi sieciami neuronowymi,
\item Keras\cite{keras} - nakładka na pakiet TensorFlow, wykorzystany w pracy w celu obróbki obrazów,
\item HyperOpt\cite{hyperopt} - optymalizacja parametrów algorytmów uczenia maszynowego.
\end{itemize}
Kolejną znaczącą zaletą języka Python jest bogata biblioteka standardowa. Do stworzenia odpowiednich programów posłużono się między innymi modułami odpowiedzialnymi za wielowątkowość czy logowanie zdarzeń mających miejsce w trakcie wykonania programu. Obok języka Python wykorzystany został bash, który~jest najpopularniejszą powłoką sytemu UNIX. Pozwalało to na~uruchomienie wielu skryptów jednocześnie z~różnymi parametrami (także równolegle). W~niektórych przypadkach w~celu tworzenia wykresów zamiast biblioteki Matplotlib wykorzystany został pakiet Gnuplot ze~względu na~komfort jego użycia. Całość zarzązana była przy użyciu rozproszonego systemu kontrol wersji Git.

\subsection{Środowisko}
Dzięki uprzejmości Instytutu Fizyki Jądrowej PAN w~Krakowie wszystkie obliczenia przeprowadzone zostały w~chmurze obliczeniowej CC1, co zapewniało znacznie większą moc obliczeniowej niż w~przypadku użycia komputera osobistego, a~także wygodę jeśli chodzi o~zarządzanie systemem oraz~danymi. Najmocniejsza możliwa konfiguracja dla nowo-utworzonej maszyny w~chmurze CC1 charakteryzowała się posiadaniem 12 wirtualnych procesorów oraz~19,6 GB pamięci RAM i~była, w~miarę dostępności, najczęściej używana. Obliczenia przeprowadzane zostały pod~systemem operacyjnym Linux z~użyciem dystrybucji Ubuntu. Python jest językiem przenośnym, także nie byłoby problemów użycia systemu Windows, lecz wykorzystanie systemu Linux było bardziej komfortowe z~punktu widzenia programisty.


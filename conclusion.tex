
\section{Wnioski}\label{conclusion}
Po~analizie wszystkich wyników, dotyczących poszczególnych obliczeń, należałoby odpowiedzieć na~pytanie, które~zostało zdefiniowane jako cel pracy tj. w~jakim stopniu wymienione metody mogą pomóc w~podwyższeniu skuteczności modelu w~przypadku wykorzystania uczenia maszynowego. Z~całą pewnością można stwierdzić, ze~istnieją  sytuacje, kiedy ilość danych nie jest na~tyle duża, abyśmy mogli wykorzystać uczenie maszynowe do~rozwiązania problemu. Nie da się jednak określić konkretnej liczby, która~o~tym mówi, ponieważ~zależy to zdecydowanie od~problemu, który~jest rozważany. Wszystkich przedstawionych w~pracy metod optymalizacji w~zasadzie warto użyć niezależnie od~ilości danych jaka wykorzystywana jest do~uczenia, lecz w~przypadku gdy jest ona niska, odgrywają one jeszcze większą rolę. Największa poprawa, jeśli chodzi o~skuteczność klasyfikacji, odnotowana została w~przypadku eksperymentów związanych z~wykorzystaniem ekstrakcji cech przy użyciu metod matematycznych oraz~ekstrakcji cech przy wykorzystaniu ,,wiedzy''. W~pierwszym z~nich jednak, wymóg stanowi separowalność liniowa, gdy w~celu ekstrakcji wykorzystywane jest LDA. Bardzo ogranicza to zastosowanie tej metody w~rzeczywistości ze~względu na~niezbyt często występowanie takich zbiorów w~realnym świecie. Druga z~nich bardzo dobrze pokazuje jak~ważne jest dobranie odpowiednich atrybutów zwłaszcza w~przypadku braku dostępu do~większej ilości danych. Pozostałe dwa eksperymenty, jeden związany z~wykorzystaniem walidacji krzyżowej oraz~drugi z~powielaniem danych, nie odnotowują już tak pokaźnych wzrostów skuteczności, lecz mogą stanowić znaczną pomoc. We wszystkich przypadkach testowane były 4 metody w~różnych wariantach, co generowało bardzo duży narzut czasowy jeśli chodzi o~ich przeprowadzenie. W~przypadku głębokich sieci neuronowych, zazwyczaj uczenie odbywa się przy wykorzystaniu zespołów procesorów graficznych, aby~przyspieszyć obliczenia. Podejmowane są także próby wykorzystania bardziej wyspecjalizowanego sprzętu takiego jak~np. FPGA \footnote{FPGA - Field-Programmable Gate Array} \cite{nurvitadhi2017can}. Pokazuje to jak~bardzo ograniczonym się jest, gdy wykorzystuje się jedynie jednostki CPU w~tym zagadnieniu.

Oprócz powyższych wniosków, autor pracy może niewątpliwie podzielić się swoimi spostrzeżeniami dotyczącymi samego tworzenia oprogramowania związanymi z~praktycznymi aspektami analizy przeprowadzonej w~pracy. Nieocenioną pomocą w~przeprowadzeniu obliczeń było wykorzystanie chmury obliczeniowej CC1. Pomimo dostępu do~znacznie większej mocy obliczeniowej niż zapewnia komputer personalny, w~wielu miejscach pracy można spotkać się z~określeniem, że~pewne obliczenia były ograniczone ze~względu na~czas ich wykonania, który~wahał się od~kilkudziesięciu minut to nawet kilkudziesięciu godzin w~zależności od~zadania. Ciężko wyobrazić sobie w~takim razie, jak~bardzo przeprowadzone obliczenia byłaby ograniczone w~przypadku wykorzystywania komputera personalnego. Idąc tym tropem, podczas tworzenia tego typu systemów, szybko dochodzi się też do~wniosku, iż~współczesne rozwiązania tego typu powinny stawiać na~równoległość zwłaszcza biorąc pod~uwagę, że~moc pojedynczego procesora jest ograniczona fizycznie. Swoją rolę w~tego typu pracach odgrywa również inżyniera oprogramowania. Tworzone linijki kodu, funkcje, klasy były wielokrotnie wykorzystywane w~kilku systemach. W~związku z~tym, dobrze jest pamiętać, aby~tworzyć kod, który~jak~w~największym stopniu będzie pozwalał na~jego wykorzystanie w~innym miejscu. Często programista uświadamia sobie to dopiero po~napisaniu wszystkiego co potrzebuje, ale~z~pewnością, zwracając na~to trochę uwagi, zaoszczędzi na~tym wiele swojego czasu.


\thispagestyle{empty}
%% ------------------------ NAGLOWEK STRONY ---------------------------------
\includegraphics[height=37.5mm]{res/agh_logo.pdf}\\
\rule{30mm}{0pt}
{\large \textsf{Wydział Fizyki i~Informatyki Stosowanej}}\\
\rule{\textwidth}{3pt}\\
\rule[2ex]
{\textwidth}{1pt}\\
\vspace{7ex}
\begin{center}
{\LARGE \bf \textsf{Praca magisterska}}\\
\vspace{13ex}
% --------------------------- IMIE I NAZWISKO -------------------------------
{\bf \Large \textsf{Paweł Pęksa}}\\
\vspace{3ex}
{\sf\small Kierunek studiów:} {\bf\small \textsf{Informatyka Stosowana}}\\
\vspace{1.5ex}
{\sf\small Specjalność:} {\bf\small \textsf{Modelowanie i~Analiza Danych}}\\
\vspace{10ex}
%% ------------------------ TYTUL PRACY --------------------------------------
{\bf \huge \textsf{Optymalizacja algorytmów uczenia maszynowego dla niskich statystyk próbek danych}}\\
\vspace{14ex}
%% ------------------------ OPIEKUN PRACY ------------------------------------
{\Large Opiekun: \bf \textsf{dr hab. Marcin Wolter}}\\
\vspace{22ex}
{\large \bf \textsf{Kraków, grudzień 2017}}
\end{center}
%% =====  STRONA TYTUŁOWA PRACY MAGISTERSKIEJKIEJ ====

\newpage

%% =====  TYŁ STRONY TYTUŁOWEJ PRACY MAGISTERSKIEJKIEJ ====
{\sf Oświadczam, świadomy(-a) odpowiedzialności karnej za~poświadczenie nieprawdy, że~niniejszą pracę dyplomową wykonałem(-am) osobiście i~samodzielnie i~ nie korzystałem(-am) ze~źródeł innych niż wymienione w~pracy.}

\vspace{14ex}

\begin{center}
\begin{tabular}{lr}
~~~~~~~~~~~~~~~~~~~~~~~~~~~~~~~~~~~~~~~~~~~~~~~~~~~~~~~~~~~~~~~~~ &
................................................................. \\
~ & {\sf (czytelny podpis)}\\
\end{tabular}
\end{center}

%% =====  TYL STRONY TYTULOWEJ PRACY MAGISTERSKIEJKIEJ ====

\newpage
\rightline{Kraków, ?? grudnia 2017}
\begin{center}
{\bf Tematyka pracy magisterskiej i~praktyki dyplomowej
Pawła Pęksy,
studenta V roku studiów kierunku Informatyka Stosowana, specjalności Modelowanie i~Analiza Danych}\\
\end{center}

Temat pracy magisterskiej:
{\bf Optymalizacja algorytmów uczenia maszynowego dla niskich statystyk próbek danych }\\

\begin{tabular}{rl}

Opiekun pracy:                  & dr hab. Marcin Wolter\\
Recenzenci pracy:               & dr inż. Tomasz Bołd\\
Miejsce praktyki dyplomowej:    & IFJ PAN, Kraków\\
\end{tabular}

\begin{center}
{\bf Program pracy magisterskiej i~praktyki dyplomowej}
\end{center}

\begin{enumerate}
\item Omówienie przedmiotu pracy magisterskiej z opiekunem.
\item Wstępne zapoznanie się z literaturą oraz ramowy plan pracy.
\item Praktyka dyplomowa:
\begin{itemize}
\item przygotowanie oprogramowania i przeprowadzenie wstępnych obliczeń,
\item weryfikacja poprawności obliczeń oraz ewentualna poprawa błędów oprogramowania
\item dyskusja i analiza wyników,
\item sporządzenie sprawozdania z praktyki.
\end{itemize}
\item Kontynuacja obliczeń związanych z tematem pracy magisterskiej.
\item Opracowanie wyników obliczeń oraz przygotowanie odpowiednich tabel oraz wykresów.
\item Analiza wyników, ponowna weryfikacja oraz zatwierdzenie obliczeń przez opiekuna.
\item Opracowanie redakcyjne pracy.
\end{enumerate}


\noindent
Termin oddania w~dziekanacie: ?? grudnia 2017\\[1cm]

\begin{center}
\begin{tabular}{lcr}
.............................................................. & ~~~ &
.............................................................. \\
(podpis kierownika katedry) & & (podpis opiekuna) \\
\end{tabular}
\end{center}

\newpage

\noindent
Na~kolejnych dwóch stronach proszę dołączyć kolejno recenzje pracy popełnione przez~Opiekuna oraz~Recenzenta (wydrukowane z~systemu MISIO i~podpisane przez~odpowiednio Opiekuna i~Recenzenta pracy). Papierową wersję pracy (zawierającą podpisane recenzje) proszę złożyć w~dziekanacie celem rejestracji co najmniej na~tydzień przed planowaną obroną.

\linespread{1.3}
\selectfont
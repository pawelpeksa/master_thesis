\section{Przeprowadzone eksperymenty} \label{results}


\subsection{Sztuczne powiekszanie zbioru danych}
\subsubsection{Opis problemu}
Eksperyment sztucznego powiększania danych został przeprowadzony w opraciu o konkurs ,,Dogs vs. Cats'' opbulikowanego w serwisie Kaggle.\footnotemark Zadaniem uczestników było stworzenie systemu, który jest w stanie rozwiązać problem rozpoznawania psów i kotów na obrazie. Udostępnione zostało baza danych 25000 sklasyfikowanych obrazów. Mały wycinek udostępnionych danych został przedstawiony na rysunku \ref{kaggle}. W celu ewaluacji i otrzymania wyniku wykorzystywany był zbiór 12500 niesklasyfikowanych obrazów. Najlepsze rezultaty uczestnicy otrzymywali przy użyciu głębokich sieci neuronów i także ta metoda została wykorzystana w tej pracy.Celem tego eksperymentu było zbadania poprawy klasyfikacji przy użyciu sztucznego powielania danych.
\footnotetext{\url{https://www.kaggle.com/c/dogs-vs-cats}}
\subsubsection{Stworzony system}
\subsubsection{Wyniki}
\subsubsection{Wnioski}

\subsection{Walidacja krzyżowa}
\subsubsection{Opis problemu}
\subsubsection{Wyniki}
\subsubsection{Wnioski}

\subsection{Ekstrakcja cech przy użyciu metod matematycznych}
\subsubsection{Opis problemu}
\subsubsection{Stworzony system}
\subsubsection{Wyniki}
\subsubsection{Wnioski}

\subsection{Ekstrakcja cech przy użyciu ,,wiedzy''}
\subsubsection{Opis problemu}
\subsubsection{Stworzony system}
\subsubsection{Wyniki}
\subsubsection{Wnioski}

